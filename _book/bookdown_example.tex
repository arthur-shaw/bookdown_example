% Options for packages loaded elsewhere
\PassOptionsToPackage{unicode}{hyperref}
\PassOptionsToPackage{hyphens}{url}
%
\documentclass[
]{book}
\usepackage{amsmath,amssymb}
\usepackage{lmodern}
\usepackage{ifxetex,ifluatex}
\ifnum 0\ifxetex 1\fi\ifluatex 1\fi=0 % if pdftex
  \usepackage[T1]{fontenc}
  \usepackage[utf8]{inputenc}
  \usepackage{textcomp} % provide euro and other symbols
\else % if luatex or xetex
  \usepackage{unicode-math}
  \defaultfontfeatures{Scale=MatchLowercase}
  \defaultfontfeatures[\rmfamily]{Ligatures=TeX,Scale=1}
\fi
% Use upquote if available, for straight quotes in verbatim environments
\IfFileExists{upquote.sty}{\usepackage{upquote}}{}
\IfFileExists{microtype.sty}{% use microtype if available
  \usepackage[]{microtype}
  \UseMicrotypeSet[protrusion]{basicmath} % disable protrusion for tt fonts
}{}
\makeatletter
\@ifundefined{KOMAClassName}{% if non-KOMA class
  \IfFileExists{parskip.sty}{%
    \usepackage{parskip}
  }{% else
    \setlength{\parindent}{0pt}
    \setlength{\parskip}{6pt plus 2pt minus 1pt}}
}{% if KOMA class
  \KOMAoptions{parskip=half}}
\makeatother
\usepackage{xcolor}
\IfFileExists{xurl.sty}{\usepackage{xurl}}{} % add URL line breaks if available
\IfFileExists{bookmark.sty}{\usepackage{bookmark}}{\usepackage{hyperref}}
\hypersetup{
  pdftitle={A Minimal Book Example},
  pdfauthor={Yihui Xie},
  hidelinks,
  pdfcreator={LaTeX via pandoc}}
\urlstyle{same} % disable monospaced font for URLs
\usepackage{longtable,booktabs,array}
\usepackage{calc} % for calculating minipage widths
% Correct order of tables after \paragraph or \subparagraph
\usepackage{etoolbox}
\makeatletter
\patchcmd\longtable{\par}{\if@noskipsec\mbox{}\fi\par}{}{}
\makeatother
% Allow footnotes in longtable head/foot
\IfFileExists{footnotehyper.sty}{\usepackage{footnotehyper}}{\usepackage{footnote}}
\makesavenoteenv{longtable}
\usepackage{graphicx}
\makeatletter
\def\maxwidth{\ifdim\Gin@nat@width>\linewidth\linewidth\else\Gin@nat@width\fi}
\def\maxheight{\ifdim\Gin@nat@height>\textheight\textheight\else\Gin@nat@height\fi}
\makeatother
% Scale images if necessary, so that they will not overflow the page
% margins by default, and it is still possible to overwrite the defaults
% using explicit options in \includegraphics[width, height, ...]{}
\setkeys{Gin}{width=\maxwidth,height=\maxheight,keepaspectratio}
% Set default figure placement to htbp
\makeatletter
\def\fps@figure{htbp}
\makeatother
\setlength{\emergencystretch}{3em} % prevent overfull lines
\providecommand{\tightlist}{%
  \setlength{\itemsep}{0pt}\setlength{\parskip}{0pt}}
\setcounter{secnumdepth}{5}
\usepackage{booktabs}
\ifluatex
  \usepackage{selnolig}  % disable illegal ligatures
\fi
\usepackage[]{natbib}
\bibliographystyle{apalike}

\title{A Minimal Book Example}
\author{Yihui Xie}
\date{2021-04-19}

\begin{document}
\maketitle

{
\setcounter{tocdepth}{1}
\tableofcontents
}
\hypertarget{preface}{%
\chapter*{Preface}\label{preface}}
\addcontentsline{toc}{chapter}{Preface}

The measurement of food consumption and expenditure is a fundamental component of any analysis of poverty and food security,
and hence the importance and timeliness of devoting attention to the topic cannot be overemphasized as the international
development community confronts the challenges of monitoring progress in implementing the 2030 Agenda for Sustainable
Development.

In 2014, the International Household Survey Network published a desk review of the reliability and relevance of survey questions
as included in 100 household surveys from low- and middle-income countries. The report was presented in March 2014
at the forty-fifth session of the United Nations Statistical Commission (UNSC), in a seminar organized by the Inter-Agency and
Expert Group on Food Security, Agricultural and Rural Statistics (IAEG-AG).

The assessment painted a bleak picture in terms of heterogeneity in survey design and overall relevance and reliability of the
data being collected. On the positive side, it pointed to many areas in which even marginal changes to survey and questionnaire
design could lead to a significant increase in reliability and consequently, great improvements in measurement accuracy.
The report, which sparked a lot of interest from development partners and UNSC member countries, prompted IAEG-AG to
pursue this area of work with the ultimate objective of developing, validating, and promoting scalable standards for the measurement
of food consumption in household surveys.

The work started with an expert workshop that took place in Rome in November 2014. Successive versions of the guidelines
were drafted and discussed at various IAEG-AG meetings, and in another expert workshop organized in November 2016 in
Rome. The guidelines were put together by a joint FAO-World Bank team, with inputs and comments received from representatives
of national statistical offices, international organizations, survey practitioners, academics, and experts in different disciplines
(statistics, economics, nutrition, food security, and analysis). A list of the main contributors is included in the acknowledgment
section. In December 2017 a draft of the guidelines was circulated to 148 National Statistical Offices from low- to high-income
countries for comments. The document was revised following that consultation and submitted to UNSC which endorsed it at
its forty-ninth session in March 2018 under item 3(j) of the agenda, agricultural and rural statistics. The version presented here
reflects what was endorsed by the Commission, edited for language. The process received support from the Global Strategy
for Agricultural and Rural Statistics.

The document is intended to be a reference document for National Statistical Offices, survey practitioners, and national and
international agencies designing household surveys that involve the collection of food consumption and expenditure data.

\hypertarget{intro}{%
\chapter{Introduction}\label{intro}}

\hypertarget{background-and-motivation}{%
\section{Background and Motivation}\label{background-and-motivation}}

Food is an important component of many fundamental dimensions of welfare, such as food security, nutrition, and health. It comprises the largest share of total household expenditure in low-income countries, accounting for about 50 percent of the average household budget (USDA, 2011) and accordingly.it is key for consumption and poverty analysis. Low levels of food access play a role in explaining why around 815 million individuals were estimated to be chronically undernourished in 2016 (FAO, WFP, IFAD, UNICEF, and WHO, 2017). Data on food consumption and expenditure underpin the most widely used measures of poverty and of food security. The collection of high-quality food consumption data is, therefore, central to the assessment and monitoring of the well-being of any human population, and is of interest to governments, international agencies, and anyone concerned with monitoring and understanding trends in social, economic, and human development.

\textbf{BOX 1 --- THE CONCEPTS OF FOOD CONSUMPTION}
Food data collected in HCES can be diverse, and often refer to diverse concepts. Even the term ``food consumption'' lends itself to multiple meanings. When the focus of the analysis is expenditure, the term ``consumption'' can designate the purchase of foods, disregarding the end-use of what was purchased. At the opposite end, analyses and surveys that focus on nutrition use the term ``food consumption'' to designate the intake of a food, possibly net of unusable parts. Throughout this document the term ``food consumption'' is used in a general sense and encompasses concepts or data that include food consumption, acquisition, expenditure, and intake. Additional descriptive are specifically used in places where their specific meanings are addressed or contrasted, or for details that relate to that precise terminology

Data on food consumption are needed, for example, to build the indicators and monitor some of the targets set for Sustainable Development Goals 1 and 2 (ending poverty and hunger). Similarly, data on food consumption are needed to assess and guide the mandate of FAO to help eradicate hunger, food insecurity, and malnutrition and the twin goals of the World Bank to eliminate extreme poverty and boost shared prosperity.\footnote{For a list of indicators that can be derived from food data collected in HCES, see Moltedo et al.~(2014); Foster et al.~(2013).} Even more importantly, national and local governments and non-governmental organizations need high-quality food consumption data to guide local and regional analysis and policy, as the mismeasurement of food consumption can lead to the misallocation of funds and may compromise the design, monitoring, and evaluation of relevant policies and programs.

\ldots and so on\ldots{}

\hypertarget{objectives-and-audience}{%
\section{Objectives and audience}\label{objectives-and-audience}}

A product of a consultation process within the international statistical community, which culminated in the endorsement by UNSC member countries at its 49th Session, these guidelines are intended to be a reference document for national statistical offices and survey practitioners designing household consumption and expenditure surveys. In putting forward these scalable standards the IAEG-AG also seeks to promote an increase in the harmonization of survey instruments and the comparability of the resulting data. Against the backdrop of this institutional context, the guidelines have multiple aims:

First, they will provide survey practitioners tasked with designing and implementing HCES in low-income settings with a harmonized set of guiding principles. The aim is to inform the main decisions that need to be taken when designing HCES, factoring in the objective of serving a wide range of users, without compromising data quality.

Second, by putting forward a vision for some of these principles, the guidelines can serve as the basis for an international dialogue between practitioners and data users coming from different disciplines and looking for different features in the data.

Third, a set of guidelines that can be widely shared and agreed upon will increase the harmonization of the surveys that are implemented (and the resulting data) and give greater coherence to the advice that national statistical offices receive from the international statistical community. Often, different users and institutions head in different directions, resulting in countries adopting very different approaches. Resulting survey design can end up reflecting priorities of donors rather than those of countries and detract from the comparability of data across countries and with other surveys within the same country.

Fourth, by identifying areas in which the consensus is based on a limited evidence-base, the guidelines can be used to chart the way for an internationally agreed survey methodology research agenda. Importantly, the guidelines can be an entry point for sustaining an interdisciplinary dialogue for the advancement of this agenda, which can bring together statisticians, economists, food security analysts, and nutritionists to contribute to an effective repurposing of HCES that can increase the surveys' ``value for money.''

\ldots and so forth\ldots{}

\hypertarget{literature}{%
\chapter{Literature}\label{literature}}

Here is a review of existing methods.

\hypertarget{methods}{%
\chapter{Methods}\label{methods}}

We describe our methods in this chapter.

\hypertarget{applications}{%
\chapter{Applications}\label{applications}}

Some \emph{significant} applications are demonstrated in this chapter.

\hypertarget{example-one}{%
\section{Example one}\label{example-one}}

\hypertarget{example-two}{%
\section{Example two}\label{example-two}}

\hypertarget{final-words}{%
\chapter{Final Words}\label{final-words}}

We have finished a nice book.

  \bibliography{book.bib,packages.bib}

\end{document}
